%%%%%%%%%%%%%%%%%%%%%%%%%%%%%%%%%%%%%%%%%
%
% 
%
%%%%%%%%%%%%%%%%%%%%%%%%%%%%%%%%%%%%%%%%%

%----------------------------------------------------------------------------------------
%	PACKAGES AND OTHER DOCUMENT CONFIGURATIONS
%----------------------------------------------------------------------------------------

\documentclass[paper=a4, fontsize=11pt]{scrartcl} % A4 paper and 11pt font size
\usepackage[]{algorithm2e} % Used for loading the algorithm package
%\usepackage{hyperref} %used for embedding url
\usepackage[T1]{fontenc} % Use 8-bit encoding that has 256 glyphs
\usepackage{fourier} % Use the Adobe Utopia font for the document - comment this line to return to the LaTeX default
\usepackage[english]{babel} % English language/hyphenation
\usepackage{amsmath,amsfonts,amsthm} % Math packages

\usepackage{lipsum} % Used for inserting dummy 'Lorem ipsum' text into the template

\usepackage{sectsty} % Allows customizing section commands
\allsectionsfont{\centering \normalfont\scshape} % Make all sections centered, the default font and small caps

\usepackage{fancyhdr} % Custom headers and footers
\pagestyle{fancyplain} % Makes all pages in the document conform to the custom headers and footers
\fancyhead{} % No page header - if you want one, create it in the same way as the footers below
\fancyfoot[L]{} % Empty left footer
\fancyfoot[C]{} % Empty center footer
\fancyfoot[R]{\thepage} % Page numbering for right footer
\renewcommand{\headrulewidth}{0pt} % Remove header underlines
\renewcommand{\footrulewidth}{0pt} % Remove footer underlines
\setlength{\headheight}{13.6pt} % Customize the height of the header

\numberwithin{equation}{section} % Number equations within sections (i.e. 1.1, 1.2, 2.1, 2.2 instead of 1, 2, 3, 4)
\numberwithin{figure}{section} % Number figures within sections (i.e. 1.1, 1.2, 2.1, 2.2 instead of 1, 2, 3, 4)
\numberwithin{table}{section} % Number tables within sections (i.e. 1.1, 1.2, 2.1, 2.2 instead of 1, 2, 3, 4)

\setlength\parindent{0pt} % Removes all indentation from paragraphs - comment this line for an assignment with lots of text

%----------------------------------------------------------------------------------------
%	TITLE SECTION
%----------------------------------------------------------------------------------------

\newcommand{\horrule}[1]{\rule{\linewidth}{#1}} % Create horizontal rule command with 1 argument of height

\title{	
\normalfont \normalsize 
\textsc{Indian Institute of Technology, Kanpur, CSE} \\ [25pt] % Your university, school and/or department name(s)
\horrule{0.5pt} \\[0.4cm] % Thin top horizontal rule
\huge CS698D, Assignment 1  \\ % The assignment title
\horrule{2pt} \\[0.5cm] % Thick bottom horizontal rule
}

\author{Harshit Maheshwari} % Your name

\date{\normalsize\today} % Today's date or a custom date

\begin{document}

\maketitle % Print the title

%------------------------------------------------
%Question 1
%------------------------------------------------
\section{Problem 1}
The question is done according to the algorithm described in notes.
\begin{algorithm}
	\KwData{A sequence of words and numbers in an array: A}
	\KwResult{Decompressed sequence of words: in $output$}
	\While {($A[i] \neq \phi$)}{
		\eIf{$A(i+1).type == word$}{
			$output += A(i+1);$\\
			$arr.first = A(i+1);$\\
			$i += 2;$\\
		}{
			\If{$i>arr.length()$}{
				$return\; \text{Error}$\\
			}
			
			$word = arr(i);$\\
			$output += word;$\\
			$arr.remove(i);$\\
			$arr.begin = word;$\\
			$i++$;
		}
	}
			
\end{algorithm}
%------------------------------------------------
%Question 2
%------------------------------------------------
\section{Problem 2}
Expected Length of first run: 
\begin{align*}
\begin{split}
	E_l(2) 	&= \lim_{n\rightarrow \infty}\left(\sum_{l=1}^{n}p(l)l\right)\\
			&= 1\left(\frac{1}{2^2} + \frac{1}{2^2}\right) + 
	2\left(\frac{1}{2^3} + \frac{1}{2^3}\right) + ... \infty\\
			&= 1.\left(\frac{1}{2}\right) + 2\left(\frac{1}{2^2}\right)+... + i\left(\frac{1}{2^i}\right)...\infty\\
	\frac{E_1(2)}{2} &= 1.\left(\frac{1}{2^2}\right) + 2\left(\frac{1}{2^3}\right)+... i\left(\frac{1}{2^{(i+1)}}\right)...\ \infty\\
	&\text{Subtracting the above 2 equations:}\\
	\frac{E_1(2)}{2} &= \frac{1}{2} + \frac{1}{2^2} + \frac{1}{2^3}...\\
	&= 1\\
	E_1(2) &= 2\\
\end{split}
\end{align*}
For alphabet os size: s\\
\begin{align*}
\begin{split}
E_l(s) &= 1\left(\frac{s(s-1)}{s^2}\right) + 2\left(\frac{s(s-1)}{s^3}\right) + 3\left(\frac{s(s-1)}{s^4}\right) +\ ... \infty\\
\frac{E_l(s)}{s} &= 1\left(\frac{s(s-1)}{s^3}\right) + 2\left(\frac{s(s-1)}{s^4}\right) + 3\left(\frac{s(s-1)}{s^5}\right) +\ ... \infty\\
&\text{Subtracting the above 2 equations:}\\
E_l(s)\left(1-\frac{1}{s}\right) &= s(s-1)\left(\frac{1}{s^2} + \frac{1}{s^3} + ...\right)\\
&= \frac{s(s-1)}{s^2}\left(\frac{1}{1-\frac{1}{s}}\right)\\
E_l(s) &= \frac{s}{s-1}\\\
\end{split}
\end{align*}
%------------------------------------------------
%Question 3
%------------------------------------------------
\section{Problem 3}
We have a sequence of words: $w_1, w_2, w_3 .. w_n$\\
\textbf{To show: }\\
\begin{equation}
	l_p(w_{i-1}, w_i) \geq l_p(w_i, w_j) \forall j<(i-1)
\end{equation}
\textbf{Proof by contradiction: }\\
Suppose, $\exists j<i-1$ s.t. $l_p(w_{i-1}, w_i)[=l_1] \geq l_p(w_i, w_j) [=l_2]$\\
Then, $w_{i-1}, w_j$ match till $l_1$ and $w_{i-1}(l_1+1) \neq w_j(l_1+1)$\\
\begin{align*}
w_{i-1}(l_1+1) & \leq w_i(l_1+1)\; \text{[Lexicographically sorted]}\\
w_{i-1}(l_1+1) & \leq w_j(l_1+1)\\
w_{i-1}(l_1+1) & < w_j(l_1+1)
\end{align*}
But then $j> i-1$\\
$\Rightarrow\Leftarrow$\\
Therefore, 
	$l_p(w_{i-1}, w_i) \geq l_p(w_i, w_j) \forall j<(i-1)$
%------------------------------------------------
%Question 4
%-----------------------------------------------
\section{Problem 4}	
\textbf{To prove}\\
\begin{align*}
H(Y|X)=0\;\leftrightarrow \;Y(X)\;\text{or}
\\\forall x_0\in X\;\exists y_0 \in Y \; \text{s.t.} P(y_0|x_0)= 1 \text{and} \\
P(y_1|x_0)=0\;\forall y_1 \neq y_0
\end{align*}
\textbf{Proof}\\
$\Rightarrow$
\begin{equation}
\sum_y\sum_x P(x,y)\log \frac{1}{P(y|x)} = 0\\
\end{equation}
Since, each of the term in the summation is greater than 0, each term has to individually 0. Now, suppose,
\begin{align*}
	\exists y_1 \text{s.t.} P(y_1|x)\in(0,1)\\
	\exists y_2 \text{s.t.} P(y_2|x)\in(0,1)
\end{align*}
Then, $P(x,y_1)\log \frac{1}{P(y_1|x)} \neq 0$.\\ Also, $P(x,y_2)\log \frac{1}{P(y_2|x)} \neq 0$\\
$\Rightarrow\Leftarrow$\\
$\Leftarrow$\\
Suppose, $\exists y_0\; \text{s.t.}\; Pr(y_0|x_0)=1\; \text{and}\; Pr(y_1|x_0)=0\forall y\neq y_0$\\
Then, it is easy to see that all the terms in the equation 4.2 are 0.\\
Hence, proved. 
%------------------------------------------------
%Question 5
%-----------------------------------------------
\section{Problem 5}	
\subsection{}
\textbf{Case 1:}Sequence is unbounded. \\
Then trivially limit is $\infty$ because the sequence is monotonically deceasing.\\
\textbf{Case 2:}Sequence is bounded.\\
Let $x = \inf \ x_n$\\
$\forall \epsilon >0 \exists x_n \text{ s.t. } (x+\epsilon)>x_n$\ [By $\inf$ definition]\\
%Let $N=n$\\
$\forall m>n \ x_m<x_n$\\
$\forall m>n \ x_m<x+\epsilon$\\
$\forall \epsilon \forall m>N \ |x_m - x| < \epsilon$\\
So, $x_m \rightarrow x$\\
\subsection{}
Let the sequence be $x_1, x_2, x_3.. $\\
\begin{align*}
lim\; \text{sup}\; x = lim_{n\rightarrow \infty}\text{sup}_{m>n}x_m\\
\text{sup}_{m>i}x_m\geq \text{sup}_{m>(i+1)}x_m\\
\end{align*}
So, $\text{sup}_{m>i}x_m,\text{sup}_{m>(i+1)}x_m,...$ form a decreasing sequence. By analogy of previous result increasing sequence of real numbers has a limit. \\
Similarly,\\ 
\begin{align*}
lim\; \text{inf}\; x = lim_{n\rightarrow \infty}\text{inf}_{m>n}x_m\\
\text{inf}_{m>i}x_m\leq \text{inf}_{m>(i+1)}x_m
\end{align*}
So, $\text{inf}_{m>i}x_m,\text{inf}_{m>(i+1)x_m},...$ form an increasing sequence. By previous result this forms a decreasing sequence of real numbers has a limit. 
%------------------------------------------------
%Question 6
%------------------------------------------------
\section{Problem 6}
We will show, \\
\begin{equation}
	f=O(g)\leftrightarrow \lim sup_{n\rightarrow \infty}\frac{|f(n)|}{|g(n)|}<\infty
\end{equation}
$\Rightarrow$\\
If $g(n)=0$ then trivially $f(n)=0$. If $g(n)\neq 0:$\\
$\exists C \exists n>N \; |f(n)|<C|g(n)|$\\
\begin{equation}
\frac{|f(n)|}{|g(n)|}\leq C
\end{equation}
If a sequence: $x$ is bounded above by C then $\lim sup x\leq C$\\
Since, the sequence is bounded above by C. So, 
\begin{equation}
	\lim_{n\rightarrow\infty}sup \frac{|f(n)|}{|g(n)|}\leq C
\end{equation}
$\Leftarrow$\\
We say that $\alpha = \lim sup_{n\rightarrow \infty }f(n)$ if following conditions hold: 
\begin{align*}
	(\epsilon>0)(\exists N)(\forall n>N)(f(n)<\alpha+\epsilon)\\
	(\epsilon>0)(\forall N)(\exists n>N)(f(n)>\alpha-\epsilon)
\end{align*}
Putting $\epsilon = 1$ in the above definition. \\
\begin{equation}
	\forall n>N \frac{|f(n)|}{|g(n)|}<C+1
\end{equation}
\begin{equation}
	\forall n>N\; |f(n)|\leq(C+1)|g(n)|
\end{equation}
So, $f(n) = O(g(n))$. Hence, proved. \\
Reference: parc.im.pwr.wroc.pl/~cichon/Math/BigO.pdf
	
%------------------------------------------------
%Question 7
%------------------------------------------------
\section{Problem 7}
The function $f(x) = \frac{1}{x}$ is undefined on $x=0$ so it is discontinuous. [
\begin{align*}
	\lim_{x\rightarrow 0^-} f(x) = -\infty \\
	\lim_{x\rightarrow 0+} f(x) = +\infty \\	
\end{align*} 
Since both the limits are different we cannot define $f(x)$ at $x=0$ s.t. the function becomes continuous.]\\
$f:\mathbb{R} \rightarrow \mathbb{R}$ is continuous $\leftrightarrow \ \forall \epsilon ,x \exists \delta \forall y (|x-y| < \delta) \ \rightarrow \ (|f(x)-f(y)| < \epsilon)$\\
Consider some point $x\neq 0$ at which we will show $f(x)$ is continuous. If $y>\frac{x}{2}$\\
Choose $\delta<\text{min}\left(\frac{|x|}{2},\frac{\epsilon x^2}{2}\right)$\\
\begin{align*}
\Big|\frac{1}{x} - \frac{1}{y}\Big| = \Big|\frac{y-x}{xy}\Big| \leq \frac{\delta}{|x|(|x| - \delta)}\leq \frac{2\delta}{x^2}\leq \epsilon\\
\end{align*}
If $x\neq 0$ then $\forall \epsilon$ we can find a suitable value of $\delta$ s.t. the continuity condition holds. \\
At $x=0$ the function is discontinuous. 
%------------------------------------------------
%Question 8
%------------------------------------------------
\section{Problem 8}
Given: \\
Continuous function $f:[0,1] \rightarrow \mathbb{R}$
Since, $f$ is continuous on $[0,1]$, \\
$\forall \epsilon ,x \exists \delta \forall y\ \ |x-y| < \delta\  \rightarrow\ |f(x) - f(y)| < \epsilon ,\ x,y \in [0,1]$\\
Let the partitions of the interval $[0,1]$ be denoted by $x_1, x_2, ... x_n$ s.t. $|x_{i+1} - x_i| < \delta\ \forall i\in [n]$ \\
For partition $P_i$ \\
\begin{align*}
\begin{split}
	U_i &= f(x_{i+1})(x_{i+1} - x_i)\\
	L_i &= f(x_i)(x_{i+1} - x_i)\\
	U_i - L_i &= (f(x_{i+1})-f(x_i))(x_{i+1} - x_i)\\
	&\leqslant \epsilon(x_{i+1} - x_i)\\
	\sum_{i=1}^{i=n} ( U_i - L_i) s&= \epsilon (1-0)\\ 
\end{split}
\end{align*}
$\Rightarrow$ $f$ is integrable.

%----------------------------------------------------------------------------------------
%	PROBLEM 9
%----------------------------------------------------------------------------------------

\section{Problem 9}

%\lipsum[2] % Dummy text
\begin{align*}
\begin{split}
L^1 &= 0\\
L^2 &= \frac{1}{2^3}(0^2 + 1^2)\\
L^3 &= \frac{1}{3^3}(0^2 + 1^2 + 2^2)\\
L^4 &= \frac{1}{4^3}(0^2 + 1^2 + 2^2 + 3^2)\\
L^5 &= \frac{1}{5^3}(0^2 + 1^2 + 2^2 + 3^2  +4^2)\\
L^n &= \frac{1}{n^3}(0^2 + 1^2 + ... + (n-1)^2)\\
& =\frac{(2n-1)n(n-1)}{6n^3}\\
L^n_{n\rightarrow\infty} &= \frac{1}{3}
\end{split}
\end{align*}

Now, checking for upper limit: \\
\begin{align*}
\begin{split}
U^1 &= 1\\
U^2 &= \frac{(1^2 + 2^2)}{2^3}\\
U^3 &= \frac{(1^2 + 2^2 + 3^2)}{3^3}\\
U^n &= \frac{(2n+1)(n+1)n}{6n^3}\\
U^n_{n\rightarrow\infty} &= \frac{1}{3}\\
\end{split}
\end{align*}
Since, $L^n_{n\rightarrow\infty} = U^n_{n\rightarrow\infty}$
the function is Reinmann integrable and the value of the integral is: $\frac{1}{3}$

%----------------------------------------------------------------------------------------
%	PROBLEM 10
%----------------------------------------------------------------------------------------
\section{Problem 10}
\textbf{To prove:}\\
\begin{equation}
	\mu (A_1\bigtriangleup A_2)=0\; \rightarrow \; \mu(A_1)=\mu(A_2)
\end{equation}
\textbf{Proof:}\\
\begin{align*}
	\mu(A_1\bigtriangleup A_2)=0\\
	\Rightarrow \mu(A_1\setminus A_2\cup  A_2\setminus 	A_1)=0\\
		\text{Since both the sets ar disjoint}\\
	\Rightarrow \mu(A_1\setminus A_2) + \mu(A_2\setminus A_1)=0\\
\end{align*}
Since the range of $\mu$ is $\mathbb{R}^+\cup 0$, both the terms have to be individually 0.\\
\begin{align*}
	\Rightarrow \mu(A_1\setminus A_2) = 0\\
	\text{Also,}\; \mu(A_1) = \mu(A_1\setminus A_2 \cup A_1\cap A_2)\\
	\mu(A_1) = \mu(A_1\setminus A_2) + \mu(A_1\cap A_2)\\
	\text{Since both the sets ar disjoint}\\
	\mu(A_1) = \mu(A_1\cap A_2)\\
	\text{Similarly,}\; \mu(A_2) = \mu(A_1\cap A_2)\\
\end{align*}
Therefore, $\mu(A_1) = \mu(A_2)$
\end{document}