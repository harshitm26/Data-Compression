%%%%%%%%%%%%%%%%%%%%%%%%%%%%%%%%%%%%%%%%%
%
% 
%
%%%%%%%%%%%%%%%%%%%%%%%%%%%%%%%%%%%%%%%%%

%----------------------------------------------------------------------------------------
%	PACKAGES AND OTHER DOCUMENT CONFIGURATIONS
%----------------------------------------------------------------------------------------

\documentclass[paper=a4, fontsize=11pt]{scrartcl} % A4 paper and 11pt font size
\newtheorem{theorem}{Theorem}
\usepackage{tabu}% Used for ready to use table properties
\usepackage[]{algorithm2e} % Used for loading the algorithm package
%\usepackage{hyperref} %used for embedding url
\usepackage[T1]{fontenc} % Use 8-bit encoding that has 256 glyphs
\usepackage{fourier} % Use the Adobe Utopia font for the document - comment this line to return to the LaTeX default
\usepackage[english]{babel} % English language/hyphenation
\usepackage{amsmath,amsfonts,amsthm} % Math packages

\usepackage{lipsum} % Used for inserting dummy 'Lorem ipsum' text into the template

\usepackage{sectsty} % Allows customizing section commands
\allsectionsfont{\centering \normalfont\scshape} % Make all sections centered, the default font and small caps

\usepackage{fancyhdr} % Custom headers and footers

\usepackage{hyperref} % Used for url 
\pagestyle{fancyplain} % Makes all pages in the document conform to the custom headers and footers
\fancyhead{} % No page header - if you want one, create it in the same way as the footers below
\fancyfoot[L]{} % Empty left footer
\fancyfoot[C]{} % Empty center footer
\fancyfoot[R]{\thepage} % Page numbering for right footer
\renewcommand{\headrulewidth}{0pt} % Remove header underlines
\renewcommand{\footrulewidth}{0pt} % Remove footer underlines
\setlength{\headheight}{13.6pt} % Customize the height of the header

\numberwithin{equation}{section} % Number equations within sections (i.e. 1.1, 1.2, 2.1, 2.2 instead of 1, 2, 3, 4)
\numberwithin{figure}{section} % Number figures within sections (i.e. 1.1, 1.2, 2.1, 2.2 instead of 1, 2, 3, 4)
\numberwithin{table}{section} % Number tables within sections (i.e. 1.1, 1.2, 2.1, 2.2 instead of 1, 2, 3, 4)

\setlength\parindent{0pt} % Removes all indentation from paragraphs - comment this line for an assignment with lots of text

%----------------------------------------------------------------------------------------
%	TITLE SECTION
%----------------------------------------------------------------------------------------

\newcommand{\horrule}[1]{\rule{\linewidth}{#1}} % Create horizontal rule command with 1 argument of height

\title{	
\normalfont \normalsize 
\textsc{Indian Institute of Technology, Kanpur, CSE} \\ [25pt] % Your university, school and/or department name(s)
\horrule{0.5pt} \\[0.4cm] % Thin top horizontal rule
\huge CS698D, Mid-Semester Examination  \\ % The assignment title
\horrule{2pt} \\[0.5cm] % Thick bottom horizontal rule
}

\author{Harshit Maheshwari} % Your name

\date{\normalsize\today} % Today's date or a custom date

\begin{document}

\maketitle % Print the title

%------------------------------------------------
%Question 1
%------------------------------------------------
\section{Problem 1}
Reference: 
\url{https://en.wikipedia.org/wiki/Information_theory_and_measure_theory}
Let $X$ and $Y$ be two independent fair coin flips and $Z$ be their 
$XOR$. The following table shows the corresponding probability distribution: 
\begin{table}[h]
	\caption{Probability distribution of random variables $X,Y,X$}
	\centering
	\begin{tabular}{|l|l|l|l|}
		\hline
		 x,y,z=0,0,0 & $\frac{1}{4}$  &  x,y,z=0,1,0 & 0 \\
		 x,y,z=0,0,1 & 0			  &  x,y,z=0,1,1 & $\frac{1}{4}$ \\
		 \hline
	 	 x,y,z=1,0,0 & 0			  &  x,y,z=1,1,0 & $\frac{1}{4}$ \\
		 x,y,z=1,0,1 & $\frac{1}{4}$  &  x,y,z=1,1,1 & 0 \\
		\hline
	\end{tabular}
\end{table}
For this probability distribution, \\
$		I(X;Y;Z) = H(X) - H(X|Y) - H(X|Z) + H(X|Y,Z)\\\\
		 = \sum_{x}p(x)\log(p(x)) + \sum_{x,y}p(x,y)\log(p(x|y)) + \sum_{x,z}	p(x,z)\log(p(x|z)) - \sum_{x,y,z}p(x,y,z)\log(p(x|y,z))\\\\
		 = (-p(x=0)\log p(x=0)\ -p(x=1)\log p(x=1))\ + (p(x=0,y=0)\log(p(x=0|y=0))\ +\ p(x=0,y=1)\log(p(x=0|y=1))\ +\  p(x=1,y=0)\log(p(x=1|y=0))\ +\ p(x=1,y=1)\log(p(x=1|y=1))\ +\ 
(p(x=0,z=0)\log(p(x=0|z=0))\ +\ p(x=0,z=1)\log(p(x=0|z=1))\ +\  p(x=1,z=0)\log(p(x=1|z=0))\ +\ p(x=1,z=1)\log(p(x=1|z=1))\ +\ 
(p(x=0,y=0,z=0)\log(p(x=0|y=0,z=0))\ +\ p(x=0,y=0,z=1)\log(p(x=0|y=0,z=1))\ +\ p(x=0,y=1,z=0)\log(p(x=0|y=1,z=0))\ +\ p(x=0,y=1,z=1)\log(p(x=0|y=1,z=1))\ +\ p(x=1,y=0,z=0)\log(p(x=1|y=0,z=0))\ +\ p(x=1,y=0,z=1)\log(p(x=1|y=0,z=1))\ +\ p(x=1,y=1,z=0)\log(p(x=1|y=1,z=0))\ +\ p(x=1,y=1,z=1)\log(p(x=1|y=1,z=1)))\\\\
		 =2\left(-\frac{1}{2}\log\frac{1}{2}\right) + 4\left(\frac{1}{4}\log\frac{1}{2}\right) + 4\left(\frac{1}{4}\log\frac{1}{2}\right) + 0\\
		 =-1
$ \\
%------------------------------------------------
%Question 2
%------------------------------------------------
\section{Problem 2}
Reference: Algorithm Design -John Kleinberg and Eva Tardos\\
\textbf{Given:}\\
Alphabet $A={a_1,...a_k}$\\
Statistical probability distribution $(p(a_1),..p(a_k))$\\
Empirical probability distribution $(q(a_1), ... q(a_k))$\\
where
\begin{equation}
	q(a_i) = \frac{\text{no. of occurences of} a_i \text{ in } x}{n}
\end{equation}
\textbf{Claim}\\
q is always better (\textbf{or equal})than encoding with any other p. \\
\textbf{Proof}\\
If the prefix tree generated using Huffman Algorithm corresponding to p is same as q then both have same compression. \\
If the tree is different then: \\
\begin{equation}
	\rho(q) \leq \rho(p)
\end{equation}

It is easy to see that there is an optimal prefix code, with corresponding tree $T^*$, in which the two lowest-frequency letters are assigned to leaves that are siblings in $T^*$. \\
\textbf{Algorithm:}\\
The algorithm merges the two lowest frequency letters $y^*, z^* \in S$ into a single leaf $\omega$, calls itself recursively on the smaller alphabet $S'$ (all alphabets excluding $y^*$ and $z^*$ including $\omega$), and by induction produces an optimal prefix code for $S'$, represented by a labelled binary tree $T'$. $T'$ is then extended into a tree $T$ for $S$ by attaching leaves labelled $y^*$ and $z^*$ as children of the node in $T'$ labelled $\omega$. \\

\textbf{Claim:}\\
$AB(T') = AB(T) - f_\omega$, where $AB(T')$ and $AB(T)$ are the average number of bits per alphabet to encode letters in $S'$ and $S$ respectively.\\
\textbf{Proof:}
\begin{align*}
	AB(T) = \sum_{x\in S}f_x. depth_T(x)\\
	=f_{y^*}.depth_T(y^*)+f_{z^*}.depth(x^*) + \sum_{x\neq y^*,z^*}f_x.depth_T(x)\\
	=(f_{y^*} + f_{y^*}).(1 + depth_{	T'}(\omega)) + \sum_{x\neq y^*,z^*}f_x.depth_T(x)\\
	=f_{\omega}.(1 + depth_{T'}(\omega)) + \sum_{x\neq y^*,z^*}f_x.depth_T(x)\\
	=f_{\omega} + f_{\omega}.depth_{T'}(\omega) + \sum_{x\neq y^*,z^*}f_x.depth_T(x)\\				
		=f_{\omega} + \sum_{x\in S}f_x.depth_{T'}(x)\\
		=f_\omega + AB(T')				
\end{align*}
\textbf{Proof by contradiction:}\\
Suppose tree $T$ is not optimal and $\exists Z$ which is optimal tree;i.e. $AB(Z)< AB(T)$ and $Z$ has leaves representing $y^*$ and $z^*$ as siblings.\\
If we replace the leaves $y^*$ and $z^*$ with their parent leaf $\omega$, we get a tree $Z'$ that defines a prefix code for $S'$. So, $AB(Z')= AB(Z)-f_\omega$\\
Since, Z is the optimum tree we have $AB(Z)< AB(T)$; subtracting $f_\omega$ from both sides of this inequality we get $AB(Z')< AB(T')$, which contradicts the optimality of $T'$ as a prefix code for $S'$. \\\\
So, the compression corresponding to the prefix tree generated by Huffman Algorithm is optimum. 
Any probability distribution which does not generates the same prefix tree using Huffman Algorithm as given by Huffman Algorithm on the empirical probability distribution will have lesser(or equal) compression.

%------------------------------------------------
%Question 3
%------------------------------------------------
\section{Problem 3}
\subsection{}
\textbf{Given: }\\
$\phi: A\rightarrow \{0,1,2,.. |A|-1\}$ is the encoding of letters in $A$\\
A parsed phrase $I(\text{phrase}_n)$ is encoded as: \\
$(n-i)*|A| + \phi(a)$\\ 
$=(n-\text{prev}(n))*|A| + \phi(a)$\\
where $i=\text{prev}(n)$ is some previous $i^{th}$ phrase followed by a symbol $a$.
Since, $1\leq \text{prev}(n) < n$\\
\begin{align*}
	0\leq I(x[n_{j-1}+1...,\ n_j]) \leq (n-1)*|A| + |A|-1\\
	0\leq I(x[n_{j-1}+1...,\ n_j]) \leq n|A|-1
\end{align*}
Now, length of encoding:
(for $p+1$ phrases) 
\begin{align*}
	 L = \sum_{j=1}^{p+1}L_j\\
	 =\sum_{j=1}^{p+1}\left\lceil \log(j|A|)\right\rceil\\
	 \leq \sum_{j=1}^{p+1}\log(j|A|) + 1\\
	 \leq \sum_{j=1}^{p+1}\log(2j|A|)\\
	 \leq (p+1)\log(2|A|(p+1))\\
	\rho(E, x)\leq \frac{(p+1)}{n\log(|A|)}\log(2|A|(p+1))\\
\leq \frac{c(x)+1}{n\log|A|}\log(2|A|(c(x)+1))
\end{align*}
where, $c(x)$ is the maximum number of distinct phrases that we can parse. 
\subsection{}
Consider the Champornoure sequence:\\ 0,1,0,0,0,1,1,0,1,1,0,0,0,0,0,1,0,1,0,0,1,1,1,0,0,....\\
Using Lempel-Ziv compression we get: 
\begin{table}[h]
	\caption{Lempel-Ziv encoding for Champornoure sequence}
	\centering
	\begin{tabular}{|c|c|c|}
		\hline
		\multicolumn{3}{|c|}{$\lambda$}\\
		\hline
		\textbf{i}	& \textbf{n-i} & \textbf{Next Symbol} \\
		\hline
		0	&	-1	&	0\\
		0	& 	-2	&	1\\
		1	&	-2	&	0\\
		1	&	-2	&	0\\
		1	&	-3	&	1\\
		2	&	-3	&	1\\
		2	&	-4	&	1\\
		3	&	-4	&	0\\
		3	&	-5	&	1\\
		4	&	-5	&	0\\
			&	...	&	\\		
			\hline				
	\end{tabular}
\end{table}\\
Here, $n-i > i$. More specifically, 
$n-i \geq i + N$ where $N$ is 1 or 2.
%------------------------------------------------
%Question 4
%------------------------------------------------
\section{Problem 4}
\subsection{}
Reference: \url{http://www.math.caltech.edu/~vdbult/ma108a/sola108a1.pdf}\\
\textbf{To Show: }
\begin{align*}
	\lim_{n\rightarrow\infty}sup(a_n +b_n) \leq \lim_{n\rightarrow \infty}sup\ a_n + \lim_{n\rightarrow \infty} sup\ b_n
\end{align*}
\textbf{Proof:}\\
If $a_n$ or $b_n$ is unbounded then the inequality is trivial. Otherwise, \\
$sup_{m>n}a_m + sup_{m>n}b_m \geq a_x+b_x \ \forall x>n\ $[Definition of Supremum]\\
$sup_{m>n}a_m + sup_{m>n}b_m$ is upper bound for $\{a_x + b_x | x>n\}$. \\
\textbf{Claim:}\\
\begin{equation}
	sup_{m>n}a_m + sup_{m>n}b_m \geq sup_{m>n}(a_m + b_m)
\end{equation}
\textbf{Proof (by contradiction:}\\
Suppose, \\
$	sup_{m>n}a_m + sup_{m>n}b_m \leq sup_{m>n}(a_m + b_m)$\\
Let
 \begin{align}
	a=sup\{a_m:m>n\}\\
	b=sup\{b_m:m>n\}\\
	c=sup\{a_m + b_m:m>n\}\\	
	a + b<c \ \ \text{[By our assumption]}\\
	\exists \kappa \text{s.t.} c = a + b + \kappa \\
	\forall \epsilon > 0 \exists i \textbf{s.t.} a_i + b_i + \epsilon > c \ \ \text{[Definition of Supremum]}\\
	a_i + b_i + \epsilon > a + b+ \kappa\\
	a+ b \geq a_i + b_i\ \ \text{Definition of supremum} \\
	\epsilon > \kappa \ \ \text{[From 4.8,4.9]}
\end{align}
But, since we can choose $\epsilon$ we choose it to be greater than $\kappa$.\\Therefore, contradiction. Hence, proved. \\
Taking the limit in this inequality we get: 
\begin{align*}
	\lim_{n\rightarrow\infty}sup(a_n +b_n) \leq \lim_{n\rightarrow \infty}sup\ a_n + \lim_{n\rightarrow \infty} sup\ b_n
\end{align*}
\subsection{}
Sequences for which this inequality is strict are:
$a_n=(-1)^n$ and $b_n=(-1)^{n+1}$\\
\begin{align*}
	\lim_{n\rightarrow\infty} sup\ a_n = 1\\
	\lim_{n\rightarrow\infty} sup\ b_n = 1\\
	a_n + b_n = 0\ \text{So,}\\
	\lim_{n\rightarrow\infty} sup\ (a_n + b_n) = 0\\
	1+1 > 0
\end{align*}


\end{document}